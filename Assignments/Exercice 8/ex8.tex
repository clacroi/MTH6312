\documentclass[12pt,a4paper]{article}
%%%%%%%%%%%%%%%%%%%%%%%%%%%%%%%%%%%%%%%%%%%
% write which packages is used in your text
\usepackage{amsmath}
\usepackage[amsmath]{ntheorem}
\usepackage{lastpage}
\usepackage{amsfonts}
\usepackage{amssymb}
\usepackage{graphicx}
\usepackage{hyperref}

%now you can write a text with French accents 
\usepackage[utf8]{inputenc}

%%%%%%%%%%%%%%%%%%%%%%%%%%%%%%%%%%%%%%%%%%%
% write your abbreviation of symbols here

%fill the brackets with your firstname and lastname
\def\StudentName{Corentin Lacroix} 
%fill the brackets with your matricule (poly student number)
\def\StudentMatricule{1812554}
% exercise of which week? 
\def\ExerciseNo{8}




\def\R{\mathcal{R}}% set of real numbers as \R
\def\argmax{\mathrm{argmax}} % argmax
\def\argmin{\mathrm{argmin}} % argmin
\def\T{^\top} %transpose
\def\bbeta{\boldsymbol \beta} % bold beta
%%%%%%%%%%%%%%%%%%%%%%%%%%%%%%%%%%%%%%%%%%%

% define your tags here
\theoremheaderfont{\normalfont\bfseries}
\theorembodyfont{\normalfont}
\newtheorem{exercise}{Exercise}
\newtheorem{solution}{Solution}
%%%%%%%%%%%%%%%%%%%%%%%%%%%%%%%%%%%%%%%%%%%

% define what to write in paper margins
\usepackage{fancyhdr}
\pagestyle{fancy}
\lhead{{\bf~\StudentName,~\StudentMatricule}}
\chead{}
\rhead{\emph{MTH6312: Exercise~\ExerciseNo}}
\lfoot{}
\cfoot{\thepage~/~\pageref{LastPage}}
\rfoot{}
%%%%%%%%%%%%%%%%%%%%%%%%%%%%%%%%%%%%%%%%%%%


% gives more space and expands margins
\textwidth 6.4in
\textheight 9in \oddsidemargin 0in \evensidemargin 0in \topmargin -0.5in
\renewcommand{\baselinestretch}{2} 
% this puts more space between lines, so that I can write comments in between
%%%%%%%%%%%%%%%%%%%%%%%%%%%%%%%%%%%%%%%%%%%

\begin{document}


%%%%%Generates your first page
\begin{titlepage}
\begin{center}
\textsc{\LARGE Exercise no \ExerciseNo}\\[1.5cm]
\vspace{2in}
\textsc{\Large \StudentName~\StudentMatricule}\\[0.5cm]
\textsc{MTH 6312: Méthodes Statistiques D'Apprentissage}\\[0.5cm]
\today
\end{center}
\end{titlepage}
%%%% Title Page ends here

\begin{exercise}

\end{exercise}
\begin{solution}
Here the optimization problem is to find $\beta$ that solves : 
\[\begin{cases}
     \underset{\beta, \beta_0}{min|\beta|}\\
     subject\ to\ y_i(x_i\beta + \beta_0) \geq 1, i=1,\cdots, N
     \end{cases}\]
     
But this case is very simple and : we clearly know which vectors will be the support vectors and what will be the optimal separating plane. It will be the line defined by $x = 7$. The margin will be 3 and thus, $\beta$ will be $\frac{1}{3}$ and $\beta_0,\ -\frac{7}{3}$.\\

The optimization problem admits 1 solution, we could easily check that this one is a solution.\\

The image in annex sums up the case. 
\end{solution}

\begin{exercise}

\end{exercise}
\begin{solution}
These 2 cost functions don't seem equivalent. The Gaussian process cost function is the joint likelihood and the term in $\frac{1}{2} \boldsymbol{f}^\top \boldsymbol{K}^{-1} \boldsymbol f$ is the contribution of the prior to this joint likelihood. 

Gaussian processes :\\
+ it is a fully statistical method, hence has an exact statistical interpretation\\
+ We will get exact confidence intervals for prediction.\\
+ we will be able to learn hyperparameters of the model (parameters of the Kernel K from the data, regularization parameter) notably by cross validation (because it is a statistical method)\\

SVM :\\
- geometrical basis/interpretation \\
- No possible simple exact hyperparameters optimizations because cross validation is not relevant with SVM.\\
- No exact statistical interpretation \\
\end{solution}

\begin{exercise}
\end{exercise}
\begin{solution}
Let $\boldsymbol K$ be the matrix defined by $\boldsymbol K_{ij} = K(x_i,x_j) = \phi(x_i)^\top \phi(x_j)$.\\

$\forall (i,j) \in 1,\cdots,N, \ \boldsymbol K_{ij} = \phi(x_i)^\top \phi(x_j) = \phi(x_j)^\top\phi(x_i)$ if we transpose this scalar quantity.\\

Ant trivially,  $\forall (i,j) \in 1,\cdots,N, \ \boldsymbol K_{ij} = \boldsymbol K_{ij}$, that is to say $$\boxed{\boldsymbol K \ is\ symetric}$$

$\forall i \in \ 1,\cdots,N$, $\phi(x_i) = (\phi_1(x_i), \phi_2(x_i), \cdots, \phi_q(x_i))^\top$
\end{solution}

\begin{exercise}

\end{exercise}
\begin{solution}
I had difficulties finding answers and solutions to this problem. I tried to find an optimization problem whose solution is the k-NN matrix and then tried to transform this constrained optimization problem to a simple optimization problem (without constraints).\\

Using the same notations of the question, we can easily remark that finding M the 2-NN neighbors matrix is equivalent to finding a solution to the following optimization problem :

\[\begin{cases}
    \underset{M \in \{0,1\}^{n \times n}}{min} \sum_{i=1}^n \sum_{j=1}^n \boldsymbol M_{ij} \boldsymbol W_{ij}\\
     subject\ to\ \sum_{j=1}^n \boldsymbol M_{ij} = 2 \ and\ \boldsymbol M_{ii} = 0 \ \forall i \in 1,\cdots,N\\
            \end{cases}\]\\
            
In fact, here $\sum_{j=1}^n \boldsymbol M_{ij} \boldsymbol W_{ij}$ is the diagonal element $(i,i)$ of $\boldsymbol M \boldsymbol W$ (as $\boldsymbol W$ is symmetric), that is to say the total sum of the weights between the $i^{th}$ data point to its 2-NN neighbors. This is a positive sum (as $\boldsymbol M$ can only take values in $\{0,1\}$). Thus, minimizing all these quantities is equivalent to minimize the sum of all these quantitites. This is how I got to this optimization problem for finding the exact 2-NN neighbors matrix. The constraints come from the fact that :\\
(i) By line in $\boldsymbol M$, we must have exactly 2 $1$ and 2 $0$\\
(ii) Diagonal elements must be set to $0$ as a data point can't be among his own nearest neigbhors.

We can rewrite this optimization problem with the matrix Trace function as : 

\[\begin{cases}
    \underset{M \in \{0,1\}^{n \times n}}{min} Tr(\boldsymbol M \boldsymbol W)\\
     subject\ to\ \sum_{j=1}^n \boldsymbol M_{ij} = 2 \ and\ \boldsymbol M_{ii} = 0 \ \forall i \in 1,\cdots,N\\
            \end{cases}\]\\
            
We could transform this optimization problem into an optimization problem without constraints. We should add a quantity $Q(\boldsymbol M)$ to the quantity we must optimize that satisfy the following constraints :\\
(ii) $Q(\boldsymbol M)$ is zero if all constraints on $\boldsymbol M$ are respected
(iii) $Q(\boldsymbol M)$ is at least the maximum value taken by  $Tr(\boldsymbol M \boldsymbol W) = \sum_{i=1}^n \sum_{j=1}^n \boldsymbol M_{ij} \boldsymbol W_{ij}$ (which is  $\sum_{i=1}^n \sum_{j=1}^n \boldsymbol W_{ij}$, 2 * the total sum of weights) without constraints on $\boldsymbol M$ to avoid tradeoff effects.\\

This gives us the following cost function to minimize : 

\begin{equation}
\begin{split}
L(\boldsymbol M) & = Tr(\boldsymbol M \boldsymbol W) + \sum_{i=1}^n \sum_{j=1}^n \boldsymbol W_{ij} \times (\sum_{i=1}^n(\sum_{j=1}^n \boldsymbol M_{ij} - 2)^2  + \sum_{j=1}^n \boldsymbol M_{ii} \\
 & = Tr(\boldsymbol M \boldsymbol W) + \sum_{i=1}^n \sum_{j=1}^n \boldsymbol W_{ij} \times (\sum_{i=1}^n(\sum_{j=1}^n \boldsymbol M_{ij} - 2)^2 + Tr(\boldsymbol M))
\end{split}
\end{equation}\\
- The term $Tr(\boldsymbol M)$ will force the Trace of $\hat M$ the solution of the problem to be $0$ and M to have only zeros coefficients on its diagonal.\\
- the term $\sum_{i=1}^n(\sum_{j=1}^n \boldsymbol M_{ij} - 2)^2$ will force the lines sums of $\hat M$ to be exactly zero.
We could eventually replace in this function $\sum_{i=1}^n \sum_{j=1}^n \boldsymbol W_{ij}$ by a constant since it is a problem constant and $\sum_{i=1}^n(\sum_{j=1}^n \boldsymbol M_{ij} - 2)^2$ by a trace af a more complex matrix.\\


The optimization problem for getting a symmetric 2-NN is the same than previously but with one more constraints on the $\hat M$ coefficients : 

\[\begin{cases}
    \underset{M \in \{0,1\}^{n \times n}}{min} Tr(\boldsymbol M \boldsymbol W)\\
     subject\ to\ \sum_{j=1}^n \boldsymbol M_{ij} = 2,\ \boldsymbol M_{ii} = 0 \ \forall i \in 1,\cdots,N\\
     and \ \boldsymbol M_{ij} = \boldsymbol M_{ji} \ \forall (i,j) \in 1,\cdots,N
            \end{cases}\]\\
            
Then, we get a modified cost function including that forces M to be symmetric : 

 
\begin{equation}
\begin{split}
L(\boldsymbol M) & = Tr(\boldsymbol M \boldsymbol W) + \sum_{i=1}^n \sum_{j=1}^n \boldsymbol W_{ij} \times (\sum_{i=1}^n(\sum_{j=1}^n \boldsymbol M_{ij} - 2)^2 + Tr(\boldsymbol M) + \sum_{i=1}^n\sum_{j=1}^n (M_{ij} - M_{ji})^2)
\end{split}
\end{equation}

These cost functions can lead to our desired solutions for any n and k (if $k \leq n$)...

\end{solution}

\end{document}