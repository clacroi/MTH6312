\documentclass[12pt,a4paper]{article}
%%%%%%%%%%%%%%%%%%%%%%%%%%%%%%%%%%%%%%%%%%%
% write which packages is used in your text
\usepackage{amsmath}
\usepackage[amsmath]{ntheorem}
\usepackage{lastpage}
\usepackage{amsfonts}
\usepackage{amssymb}
\usepackage{graphicx}
\usepackage{hyperref}
\usepackage{listings}

%now you can write a text with French accents 
\usepackage[utf8]{inputenc}

%%%%%%%%%%%%%%%%%%%%%%%%%%%%%%%%%%%%%%%%%%%
% write your abbreviation of symbols here

%fill the brackets with your firstname and lastname
\def\StudentName{Corentin Lacroix} 
%fill the brackets with your matricule (poly student number)
\def\StudentMatricule{1812554}
% exercise of which week? 
\def\ExerciseNo{5}




\def\R{\mathcal{R}}% set of real numbers as \R
\def\argmax{\mathrm{argmax}} % argmax
\def\argmin{\mathrm{argmin}} % argmin
\def\T{^\top} %transpose
\def\bbeta{\boldsymbol \beta} % bold beta
%%%%%%%%%%%%%%%%%%%%%%%%%%%%%%%%%%%%%%%%%%%

% define your tags here
\theoremheaderfont{\normalfont\bfseries}
\theorembodyfont{\normalfont}
\newtheorem{exercise}{Exercise}
\newtheorem{solution}{Solution}
%%%%%%%%%%%%%%%%%%%%%%%%%%%%%%%%%%%%%%%%%%%

% define what to write in paper margins
\usepackage{fancyhdr}
\pagestyle{fancy}
\lhead{{\bf~\StudentName,~\StudentMatricule}}
\chead{}
\rhead{\emph{MTH6312: Exercise~\ExerciseNo}}
\lfoot{}
\cfoot{\thepage~/~\pageref{LastPage}}
\rfoot{}
%%%%%%%%%%%%%%%%%%%%%%%%%%%%%%%%%%%%%%%%%%%


% gives more space and expands margins
\textwidth 6.4in
\textheight 9in \oddsidemargin 0in \evensidemargin 0in \topmargin -0.5in
\renewcommand{\baselinestretch}{2} 
% this puts more space between lines, so that I can write comments in between
%%%%%%%%%%%%%%%%%%%%%%%%%%%%%%%%%%%%%%%%%%%

\begin{document}


%%%%%Generates your first page
\begin{titlepage}
\begin{center}
\textsc{\LARGE Exercise no \ExerciseNo}\\[1.5cm]
\vspace{2in}
\textsc{\Large \StudentName~\StudentMatricule}\\[0.5cm]
\textsc{MTH 6312: Méthodes Statistiques D'Apprentissage}\\[0.5cm]
\today
\end{center}
\end{titlepage}
%%%% Title Page ends here


\begin{exercise}
 Show that
$\hat{\bbeta} = (\mathbf{X}\T \mathbf{X} + \lambda \mathbf{I})^{-1} \mathbf{X}\T \mathbf{y}$
is a solution to minimizing
$ S(\bbeta) = (\mathbf{y} - \mathbf{X} \bbeta)\T (\mathbf{y} - \mathbf{X} \bbeta) + \lambda \bbeta\T \bbeta$.
\end{exercise}


\begin{solution}
Here write your solution. Always box your final solution in a new line, and centred, like:
\\
$$ \Rightarrow \boxed{
 S(\bbeta) = 
\mathbf{y}\T \mathbf{y} - \mathbf{y}\T \mathbf{X} \bbeta - \bbeta\T \mathbf{X}\T \mathbf{y}
+  \bbeta^T (\mathbf{X}\T \mathbf{X} + \lambda \mathbf{I}) \bbeta.
}
$$
\vspace*{1.5cm} % you can add vertical space as much as you want

I defined \textbackslash boldsymbol \textbackslash beta as \textbackslash bbeta (see \textbackslash def section before \texttt{begin\{document\}}). Now it is much faster to write the equations that include bold beta. You may need to force \LaTeX to write the rest in a new page. \newpage %writes the rest in the new page


Put your codes and its output in verbatim, it will be printed as it is. 
\begin{verbatim}
> prostate <- data.frame(lpsa=1:97, lcavol=1:97, lweight=1:97, age=1:97, lbph=1:97, svi=1:97, lcp=1:97, gleason=1:97, pgg45=1:97)
> prostate[1] <- x[9]
> prostate[2:9] <- x[1:8]
> pairs(prostate, col="purple")
\end{verbatim}
You may add some graphs, use always postscript files and Latex2Dvi, Dvi2PS, PS2PDF compiler sequence. 

\end{solution}

\begin{exercise}

\end{exercise}
\begin{solution}

\end{solution}

\begin{exercise}
Show that ..
\end{exercise}
\begin{solution}
We have : \\

\begin{equation}
\begin{split}
A & = \frac{\pi r^2}{2} \\
 & = \frac{1}{2} \pi r^2
\end{split}
\end{equation}



\[ 
\left \{
  \begin{tabular}{ccc}
  1 & 5 & 8 \\
  0 & 2 & 4 \\
  3 & 3 & -8 
  \end{tabular}
\right \}
\]

\begin{equation}
\begin{split}
(S) & = 2 \\
	& = 3
\end{split}
\end{equation}

\[\alpha(x)=\begin{cases}
               x\\
               \frac{1}{1+e^{-kx}}\\
               \frac{e^x-e^{-x}}{e^x+e^{-x}}
            \end{cases}\]

\end{solution}

\begin{exercise}
Local polynomial regression is implemented the \texttt{loess} function in R. Load the prostate data using \texttt{R>data(prostate)} from \texttt{ElemStatLearn} package and try to predict \texttt{lcavol} using only \texttt{lcp}. Visualize your prediction as a smooth function and play with parameters \texttt{span} and \texttt{degree}. Plot different fitting functions and after seeing these variants suggest some values for \texttt{span} and \texttt{degree} that fits the data better.\\
\vspace{0.2in}
{ \texttt{Lesson to learn:} Bandwith of the kernel is more important than the form of the kernel or the polynomial order of the local regression.}
\end{exercise}

\begin{solution}
I defined a R function taking a span value and a degree value as argument that plots on 1 2D figure : \\
- data points from prostate (lcavol vs. lcp)\\
- smooth prediction function obtained by local polynomial regression. \\

The function also prints sum of squared residuals, hence train $l2$ norm. Some of corresponding images for different values of span and degree can be found in annex. \\

It seems that $(span, degree) \approx (0.6, 2)$ gives the lowest test error. But this the test error and doesn't give a good idea of how the prediction function would behave for new test data. \\
R-code : 
\begin{verbatim}
plotLocalRegression <- function(span_val, degree_val) {
    localReg <- loess(lcavol ~ lcp, data=prostate, span=span_val, degree=degree_val)
    prediction <- predict(localReg, seq(-1.3864, 2.904, 0.01))
    plot(prostate$lcp, prostate$lcavol, lwd=4,
    		   pch=19, col="gray", xlab="lcp",
    		   ylab="lcavol")
    title = paste("Local polynomial regression of lcavol vs. lcp with span=", 
    toString(span_val), "and degree=", toString(degree_val))
    title(main=title)
    lines(seq(-1.3864, 2.904, 0.01), prediction, col="blue", lwd=2.5)
    legend(1.0, 0.0, c("Data points", "Smooth prediction function"), 
           lty=c(1,1), lwd=c(4,2.5), col=c("gray", "blue"))
           
    print(sum((localReg$residuals)^2))
}
\end{verbatim}
\end{solution}

\begin{exercise}
{Read ESL page 61 to 80.} Try to understand the difference between \texttt{prcomp} and \texttt{princomp} functions in R.
\end{exercise}
\begin{solution}
It seems that the differences between these 2 function come from the fact that computation of principal components aren't done on the same matrixes. For \texttt{prcomp}, this is done on the covariance matrix of the columns while for \texttt{princomp} it can be done on the scaled correlation matrix. This gives the same eigenvectors but different eigenvalues.
\end{solution}

\begin{exercise}
Simulate some regression data with correlated attributes: $n=20$ data points from the linear model  $y_i=2x_{1i}+2x_{2i}+\varepsilon_i$, where \texttt{epsilon=rnorm(0,sd=0.1)},  \texttt{x1=rnorm(20)}, but \texttt{x2=x1 + rnorm(20,sd=0.01)}. Do not forget to fix the \texttt{set.seed()} to obtain reproducible results.  Draw the scatter plot of $x_1$ versus $x_2$.
\end{exercise}

\begin{solution}
R-code : 
\begin{verbatim}
> set.seed(0)
> x1 <- rnorm(20)
> epsilon <- rnorm(20,sd=0.1)
> x2 <- x1 + rnorm(20, sd=0.01)
> plot(x1, x2, pch=19, col="gray")
\end{verbatim}
\end{solution}

\end{document}